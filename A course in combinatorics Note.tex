%!TEX program = xelatex
\documentclass[cn,hazy,sakura,14pt,normal]{elegantnote}

\title{A course in combinatorics\ 读书笔记}

\author{Monohydroxides}

\date{\zhtoday}

\usepackage{array}

\begin{document}

\begin{sloppypar}

\maketitle

\newpage

\tableofcontents

\newpage

\setcounter{page}{1}

\section{图} \label{sec: graph}

本章主要对图的基本性质进行了介绍,如平面图、重边、自环、有向性、同构、完全图、正则图、生成子图、图的连通性、回路、多边形等。

本章介绍了图的部分基本的性质。

\begin{theorem} \label{thm: First theorem of graph theory}
    任一有限图含有偶数个奇顶点。
\end{theorem}

\begin{proof}
    图论中,所有点的度数之和为边数的两倍:
    \begin{equation}
    \sum_{x \in V(G)} \text{deg}(x) = 2 \left| E(G) \right|.
    \end{equation}
    由上式可以直接得到该定理的证明。
\end{proof}

\begin{theorem} \label{thm: Eulerian graph}
    一个无孤立顶点的有限图 $G$ 是欧拉图,当且仅当 $G$ 是连通图且每一个点都是偶顶点。
\end{theorem}

在书中,定理\ref{thm: Eulerian graph}采用了一种逐步扩展的方法,给出了一个构造性的证明,且得到了有向图 $G$ 有有向欧拉回路的充分必要条件:$G$ 是联通的且每个点的入度和出度相同。

本节的 Example 1.2 中,作者提到了一个 \href{https://en.wikipedia.org/wiki/Instant\_Insanity}{Instant Insanity Puzzle},该问题的解法将摆方块游戏与通过四个方块得到的无向图的两个无公共边(且边的编号也不相同)的 2-正则的子图联系了起来,通过相对面的颜色得到无向图的边的关系,通过边的标号刻画不同的方块,通过 2-正则的子图刻画得到的长方体的一对面,非常巧妙。

本节的 Problem 1G 中,提到了一个定理:

\begin{theorem}
    一个顶点数大于等于 2 的有限简单图至少有两个点的度数相同。
\end{theorem}

\begin{proof}
    1. 若有大于等于 $2$ 个点的度数为 $0$,那么结论成立。
    
    2. 若只有一个点的度数为 $0$,那么,其余 $n - 1$ 个点的度数均不能取$n - 1$,否则不为简单图,那么这 $n - 1$ 个节点的度数只能为 $1, 2, \dots, n - 2$ 其中之一,由抽屉原理,一定存在两个点的度数相同。

    3. 若没有点的度数为 $0$,那么,这 $n$ 个点的度数只能为 $1, 2, \dots, n - 1$ 其中之一,由抽屉原理,一定存在两个点的度数相同。
\end{proof}

\section{树}

在 \ref{sec: graph} 中,有一个结论:一个不包含简单闭合回路的连通图(也就是没有子图为多边形的图),是一个树,且我们进一步还有如下结论:

\begin{theorem}
    一个有 $n$ 个点的连通图是一个树,当且仅当该连通图有 $n - 1$ 条边。
\end{theorem}

\begin{proof}
    一个有 $n$ 个点的联通图必然有 $\left| E \right| \geq n - 1$,另一方面,若该图的边数 $\left| E \right| \geq n$,那么该图一定存在为多边形的子图,此时该图不为树,因此 $\left| E \right| \leq n - 1$,综上两方面有树的边数为 $n - 1$。
\end{proof}

此外,本节开篇即介绍了一个不是很显然的公式:

\begin{theorem} [Cayley 公式]
    $n$ 个顶点的带标号无根树有 $n ^ {n - 2}$ 棵。
\end{theorem}

此外,上述定理还可以被表示为:有 $n$ 个顶点的完全图,其生成树的总数为 $n ^ {n - 2}$。

\begin{proof}
    证法1.

    对于 $n$ 个点的完全图 $K_n$,考虑有序集 $V = \{1, 2, 3, \dots, n\}$,$T$ 为 $K_n$ 的一个生成树,考虑一个树的序列 $T_1, T_2, \dots, T_{n - 1}$ 并令 $T_1 = T$,其中 $T_i$ 有 $n - i + 1(i = 1, 2, \dots, n - 1)$ 个点,令 $x_i$ 是$T_i$ 中编号最小的度为 $1$ 的点,从 $T_i$ 中删除 $x_i$ 及边 $(x_i, y_i)$,通过这种方法可以得到 $n - i$ 个点的树 $T_{i + 1}$。

    通过该方法,可以得到一个刻画 $T$ 的序列:$\mathcal{P}(T) = (y_1, y_2, \dots, y_{n - 2})$。

    若 $\mathcal{P}$ 是从 $K_n$ 的所有生成树到可能的编码的双射,那么 $K_n$ 的生成树数目为 $n ^ {n - 2}$ 就得证了。

    下面需要证明映射 $\mathcal{P}$ 是双射。

    首先,$y_{n - 1} = n$ 是显然的。其次,$x_k, x_{k + 1}, \dots, x_{n - 1}$ 和 $n$ 是 $T_k$ 中的点,最后,$(x_i, y_i), k \leq i \leq n - 1$,是按一定的顺序得到的 $T_k$ 的边。

    因为点 $v$ 在 $(x_i, y_i),1 \leq i \leq n - 1$ 中出现 $\text{deg}_{T}(v)$ 次,且恰好一次出现在 $x_1, x_2, \dots, x_{n - 1}, y_{n - 1}$ 中,因此点 $v$ 在 $y_1, y_2, \dots, y_{n - 2}$ 中出现的次数是 $\text{deg}_{T}(v) - 1$。

    我们可以发现,$T_k$ 中度为 $1$ 的点是不在下面集合中的点:
    $$\{x_1, x_2, \dots, x_{k - 1}\} \cup \{y_k, y_{k + 1}, \dots, y_{n - 1}\}$$

    其中,$\{x_1, x_2, \dots, x_{k - 1}\}$ 中为已经在 $T_k$ 之前被去掉的点,而 $\{y_k, y_{k + 1},\dots, y_{n - 1}\}$ 中的点在剩下的点中度都不为 $1$。

    这意味着 $T_k$ 中,编号最小的度为 $1$ 的顶点 $x_k$ 是有序集 $V = \{1, 2, 3, \dots, n\}$ 中不在上面两个集合的并中的最小元素。

    因此,我们有 $x_1$ 是 $V$ 中不在 $\mathcal{P}(T)$ 中的最小元素,并且我们能够通过 $\mathcal{P}(T)$ 和 $x_1, x_2, \dots, x_{k - 1}$ 唯一确定 $x_k$,因此,$\mathcal{P}$ 是双射。
\end{proof}

\begin{proof}
    证法2.
    
    考虑从 $\{2, 3, \dots, n - 1\}$ 到 $\{1, 2, \dots, n\}$ 的一个映射 $f$,显然,这样的映射有 $n^{n - 2}$ 个,由映射构造一个有向图 $D$,$D$ 的边为 $(i, f(i)), i = 2, \dots, n - 1$。

    可以发现,$D$ 由根为 $1$ 和 $n$ 的两棵树(因为构造映射时没有 $1$ 和 $n$)和一些环上挂了若干个树构成,我们将第 $i$ 个环上的最小元素放在最右边,记为 $r_i$,将 $r_i$ 指向的元素放在最左边,记为 $l_i$。

    我们可以将 $(1, l_1), (r_1, l_2), (r_2, l_3), \dots, (r_{k - 1}, l_k), \dots, (r_{n - 1}, n)$ 连接起来,同时去掉每个 $(r_i, l_i)$,就可以得到唯一的一棵树,这样也得到了一种双射的证法。
\end{proof}

\begin{proof}
    证法3.

    多项式定理(直接推广二项式定理即可得):
    $$(x_1 + x_2 + \dots + x_k)^n = \sum \binom{n}{r_1, \dots, r_k} x_1^{r_1}x_2^{r_2} \dots x_k^{r_k}$$
    由
    $$(x_1 + x_2 + \dots + x_k)^n = (x_1 + x_2 + \dots + x_k)^{n - 1} (x_1 + x_2 + \dots + x_k)$$
    可得多项式系数的 Pascal 公式:
    $$\binom{n}{r_1, \dots, r_k} = \sum_{i = 1}^{k} \binom{n - 1}{r_1, r_2, \dots, r_i - 1, \dots, r_k}$$
    
    令 $t(n;d_1, d_2, \dots, d_n)$ 表示 $n$ 个顶点的度分别为 $d_1, d_2, \dots, d_n$ 的有标号树的个数,该值只依赖于各个 $d_i$,因此不妨令 $d_1 \geq d_2 \geq \dots \geq d_n$,那么显然有 $d_n = 1$。 
    
\end{proof}

\newpage

\nocite{*}
\printbibliography[heading=bibintoc, title=\ebibname]

\end{sloppypar}

\end{document}
