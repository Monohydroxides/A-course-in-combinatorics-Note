%!TEX program = xelatex
\documentclass[cn,hazy,sakura,14pt,normal]{elegantnote}

\title{A course in combinatorics\ 读书笔记}

\author{Monohydroxides}

\date{\zhtoday}

\usepackage{array}

\begin{document}

\maketitle

\newpage

\section{图}

本章主要对图的基本性质进行了介绍,如平面图、重边、自环、有向性、同构、完全图、正则图、生成子图、图的连通性、回路、多边形等。

本章介绍了图的部分基本的性质。

\begin{theorem} \label{thm: First theorem of graph theory}
    任一有限图含有偶数个奇顶点。
\end{theorem}

\begin{proof}
    图论中,所有点的度数之和为边数的两倍:
    \begin{equation}
    \sum_{x \in V(G)} \text{deg}(x) = 2 \left| E(G) \right|.
    \end{equation}
    由上式可以直接得到该定理的证明。
\end{proof}

\begin{theorem} \label{thm: Eulerian graph}
    一个无孤立顶点的有限图 $G$ 是欧拉图,当且仅当 $G$ 是连通图且每一个点都是偶顶点。
\end{theorem}

在书中,定理\ref{thm: Eulerian graph}采用了一种类似于逐步扩展的方法,给出了一个构造性的证明,且得到了有向图 $G$ 有有向欧拉回路的充分必要条件:$G$ 是联通的且每个点的入度和出度相同。

\nocite{*}
\printbibliography[heading=bibintoc, title=\ebibname]

\end{document}
